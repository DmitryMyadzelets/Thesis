
\chapter{Conclusions and future work}
\label{chap:conclusions}


\emph{``We are stuck with technology when what we really want is just stuff that
works'' -- Douglas Adams, The Salmon of Doubt}.

The main aim of this research was to bring the fascinating beauty of formal
mathematical approaches closer to everyday practices of industrial
development engineers in order to enhance the routine process of 
the automated systems creation.

As shown in this work, the major obstacle for application of formal methods in
the industrial systems development process seems to be a lack of appropriate
design techniques and tools. Despite the fact that the formal approaches are
generally quite mature, and in the field of Discrete Event Systems particularly,
there is still a room for improvements.
As a confirmation, one can observe that every-day design processes, decision
assistance, programming tools and languages are often outdated from the
perspective of what could be used nowadays.

To show the current situation in industry the Chapter
\ref{chap:problem_description} describes, with a help of an industrial plant
real example, what quantitative and qualitative methods are used nowadays,
besides the formal approaches, in order to achieve desired properties of
industrial systems.

A few goals were persuaded in this work. The first one is the modeling
approach for the manufacturing systems design. The effort has been taken in
the development of formal representations of widely used industrial components.
For this approach the framework of formal languages and automata has
been chosen. The choice is based on high availability of formal approaches
using this framework, which somehow confirms efficiency of these formalisms on
the one hand, and a relative simplicity of the mathematical notation and graphical
representation of automata models on the other hand, bearing in mind that the
results of the research should be understood by a wide community of industrial
engineers, with a little or no effort.

The automata design patterns, developed and validated during this research,
incorporate both nominal and faulty behaviours. This approach appeared to be
possible after decomposition of hardware components into small predefined
models. These models may efficiently abstract details of the
formal representation, and ``hide'' them from a designer of an
industrial system.
In order to increase the level of abstraction, thus enhancing usability and
reusability, the formal models are encapsulated into UML-like blocks of
Generalised Devices. This modeling process, as well as an essential mathematical
notation for discrete event systems in terms of languages and automata, and
diagnosability types are described in Chapter \ref{chap:framework}.

Use of the formal methods often corresponds with a problem of exponential
explosion which may make the a solution for the problem intractable. The
diagnosability analysis is not an exception. Approaches which rely on modular
nature of the complex system use a variety of notions and algorithmic techniques
in order to tackle the computational burden. This work has introduced a new
definition of diagnosability and a notion of virtual modules. It can be seen
as an algorithmic ``trick'' which combines the existing modules of the system
into new modules (which do not correspond to real components, and that
is why they are called ``virtual'') in a way that the system with the new
modularity is modular diagnosable. The theoretical part of this work is covered
in Chapter \ref{chap:theory}.

The final goal of the research was to focus on applicability of the developed
concepts. In order to validate the algorithms which verify diagnosability and
provide information for construction of virtual modules, a software tool has
been created. It is described in the Chapter \ref{chap:simulation}. There an
application of this tool for an instance of failure in the real
industrial process is demonstrated and compared with the centralized approach.

Concluding, modeling and theoretical approaches have been developed and then 
validated with a software tool, proving efficiency of the concepts.
Particularly, the performance advantage with respect to the centralized approach
is obvious. The advantages and disadvantages of the method with comparison to
other algorithmic techniques, even if they differ from the theoretical point of
view, has to be checked in the future.
Additional effort is also required for performance optimisations of the
developed software tool.
