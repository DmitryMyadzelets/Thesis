
\chapter*{Introduction}
\addcontentsline{toc}{chapter}{Introduction}


% \section{Introducton to Introduction}

Modern society is characterized by the growing demand for the use of automated
systems in the everyday life, and these systems require software engineering
with constantly increasing complexity. Industrial automation software
engineering has traditionally been a field with the most strict requirements
and highest standards applied. However, it is still a common practice for an
automation software design to consist of writing a ladder logic for programmable
logic controller (PLC) with partially and ambiguously determined specifications,
without clearly defined software architecture, and no formal verification of the
system.
Reflecting the importance of the software and a growing ratio of the software
cost to the costs of machinery, engineers and researches put a lot of effort
toward facilitating the development and maintenance of software, increasing its
performance and reliability while decreasing the cost of its life cycle. One of
the major established trends is use of standardised component solutions for
industrial automation systems aiming at portability, reusability,
interoperability and reconfiguration of applications. Another tendency is
the growing number of formal methods - mathematical approaches supporting
decisions making process during systems design and operations.

\section*{Design Process of Industrial Systems}

According to the standard ISO/IEC 12207 \cite{_iso/iec_12207_2008} the
industrial software development process (life cycle process) can be structured
into six stages: requirements specification, software design, implementation and
integration, testing, deployment and maintenance. Starting from the
implementation the stages mainly depend on a particular vendor of hardware
and a dedicated software - supervisory control acquisition system (SCADA).
There are hundreds of major producers of automation hardware and software, but
the variety of ways the developers can write programs is limited by the standards,
such as {IEC 61131}, making them less error prone and more reusable. In other
words, this sequence of stages is quite mature. 

The stage of requirements specification consists of analyzing, documenting and
validating the needs and conditions for technological process, as well as rules,
constrains and policies for the plant hardware. The most known standard used at
this stage is the Unified Modeling Language (UML), developed by the Object
Management Group (OMG) Technology Standards Consortium. This standard
facilitates also the next stage, software design, with a Model-Driven
Architecture (MDA) approach. The industrial software design stage adopts
numerous methods, such as component based approach, object-oriented and
aspect-oriented programming, and software product line, etc (see
\cite{vyatkin_software_2013} for an extensive survey of the state of the art for
software engineering methods in industrial automation).

\section*{Problems Statement}

The major drawback of the aforementioned industrial software
development process is that it is mainly designed for humans, whereas the
reliability and other important properties of systems in large part depend on
machine-oriented formal verification methods. Formal methods apply
mathematically-based techniques to the development of systems, from the
specification level to the implementation level. These methods proved
to be effective, especially for safety critical systems, but due to their
mathematical nature and lack of the supporting tools the use of formal methods
in industrial practice is not common yet. The problem is that mathematical
notation requires to have additional knowledge on the part of the development
engineers, creating a psychological barrier for them, and also that the 'de
facto' development process does not incorporate a formal representation.

A way to overcome the above problem is to mix formal techniques with
``standard-based'' approaches/tools which are already adopted in industry. If
a non formal (or semi-formal) systems representation has the ability to be
translated into a completely formal form, then mathematical techniques can be
applied. Examples of such approaches are \cite{dong_model_2001}, where authors
extend UML representation with a ``collaboration diagram'' and translate it to
extended hierarchical automata, \cite{secchi_use_2007} which formalizes UML
into Dirac structures, and \cite{zhou_semantic_2012} where the authors
translate Simulink diagrams into input/output extended finite automata.

Even though the translation into a mathematical representation gives the
opportunity to apply formal techniques, the requirements for the system's
properties should again be expressed in a formal way by engineers, which
raises the above mentioned problems. Thus, despite of the fact that some
successful reports can be found in literature, progress along these lines seems
minimal
% , andapplicable only with numerous constrains both for the methods and the tools
. 

Another solution to overcome the problem is the development of new approaches
which incorporate advantages of both methodologies, the one suitable for humans,
and the other using formal techniques. Nowadays, when the growing complexity of
automated systems imposes requirements which can be met only by formal methods,
the relevant response may be the creation of modeling techniques when the models
already encapsulate their formal representations. Thus, the underlining
mathematical nature would be ``hidden'' from engineers, and verification of
important properties can be performed automatically. An example of such approach
is presented in \cite{sartini_architectures_2010} where the authors propose
a library of general UML blocks where each block is corresponded to a predefined
automaton. The simple blocks can be used then to construct more complex
entities while their formal representation can be achieved automatically by
composition of the predefined automata.

The necessity of composition of modules formal representations for the sake of
verification of some important properties gives rise to another problem, which
is inseparable from the formal methods, - the problem of so-called \emph{state
explosion}. This problem can be solved by development of mathematical
techniques which efficiently exploit the modular nature of the systems such
that the corresponding computational burden remains at an acceptable level.

As soon as the quantity and quality of mathematical methods and their
implementations is sufficient for the current complexity of automated systems,
and these methods can be encapsulated into integrated development tools as a
first-class citizen, seamlessly providing the power of formal approaches through
the easy-to-understand visual modeling process, the development process of
automated systems will move to the next stage of its evolution.

\section*{Contribution}

This work aims to facilitate the process of industrial automated systems
development applying formal methods to ensure the reliability of systems. From
many existing problems, for this work has been chosen the problem of
verification of system's design in terms of diagnosability. Diagnosability
problem answers the question: either the given system is ready for fault
diagnosis or not. According to the IEC vocabulary \cite{iec_vocabulary} the
\emph{fault diagnosis} are ``actions taken for fault recognition, fault
localization and cause identification'', and a \emph{fault} is ``the state of an
item characterized by inability to perform a required function''.

To achieve the goals, this work exploits Discrete Event Systems theory
\cite{cassandras_introduction_2010} and its automata framework. The theory of
automata is chosen due to its relative simplicity. Indeed, it has just two basic
entities (state and event) and intuitively intelligible graphical
representation. Taking apart a language theory, which is necessary for
research, the automata framework does not require much additional
mathematical knowledge for automation engineers. This makes it the first
candidate as a formal mathematical tool for future general purpose integrated
development environments (IDE) in automation industry.

The main contributions are following:
\begin{itemize}
  \item A new formulation of distributed diagnosability problem. This 
  formulation is then used to enforce the desired property of the system,
  rather then just verifying it. This approach tackles the state explosion problem
  which arises when the automata framework is applied for complex discrete event
  systems.
  \item Presented new structural patterns, in terms of automata
  framework. These patterns are aimed to decrease computation burden while
  applying verification algorithms.
  \item Created algorithms for verification of diagnosability property in the
  context of the new distributed diagnosability problem.
  \item Extended the approach of \emph{GeneralizedDevice}, recently
  developed in the University of Bologna \cite{faldella_hierarchical_2008},
  \cite{sartini_architectures_2010}. This work extends it towards formalization
  of failures representation in order to apply the developed algorithms. It
  allows the embedding of a formal approach into a higher level of the
  systems' representation of ``standard'' modular design process, similar to one
  exploiting UML design tools.
  \item To validate the concepts developed during the research, a software tool
  was created. This tool exploits web-technologies in order to provide zero-time
  access to it, the possibility for easy modifications in order to solve other
  formal problems, and opportunities for further enhancement by community.
\end{itemize}


\section*{Organization}

Chapter \ref{chap:problem_description} gives an overview of the problem of
fault diagnosis in a real example of an industrial automation plant. It
provides a short description of the technological process with a couple of
illustrative failures. The types of failures are discussed, and how particular
system's failures can be determined, evaluated and treated. 

Chapter \ref{chap:framework} describes the UML based modeling framework, and how 
the framework specifies hardware modules with automata. Then the
formal framework for discrete event systems is briefly presented, different
definitions of diagnosability are explained, and the general idea of new notion
of virtual modular diagnosability is given.

Chapter \ref{chap:theory} presents the new type of diagnosability -
virtual modular diagnosability, in details. It is firstly explained with a
simple example of a system consisting of two modules, then a general form of
the approach is described. At the end of the chapter algorithms for verification
of the virtual modular diagnosability are described.

Chapter \ref{chap:simulation} introduces a new tool for modeling and simulation
with automata. An application of the tool to the example of the real system
presented in the Chapter \ref{chap:problem_description} is shown.

The conclusions for the work are given at the end. 

% \emph{Formal methods aim to apply mathematically-based techniques to the
% development of computer-based systems, especially at the specification level,
% but also down to the implementation level. This aids early detection and
% avoidance of errors through increased understanding. It is also beneficial for
% more rigorous testing coverage. This talk presents the use of formal methods on
% a real project. The Z notation has been used to specify a large-scale high
% integrity system to aid in air traffic control. The system has been implemented
% directly from the Z specification using SPARK Ada, an annotated subset of the
% Ada programming language that includes assertions and tool support for proofs.
% The Z specification has been used to direct the testing of the software through
% additional test design documents using tables and fragments of Z. In addition,
% Mathematica has been used as a test oracle for algorithmic aspects of the
% system. In summary, formal methods can be used successfully in all phases of the
% lifecycle for a large software project with suitably trained engineers, despite
% limited tool support.}
% 
% \emph{Formal methods have been advocated as a means of increasing the
% reliability of systems, especially those which are safety or business critical,
% but the industrial uptake of such methods has been slow. This is due to the
% perceived difficulty of mathematical nature of these methods, the lack of tool
% support, and the lack of precedents where formal methods has been proved to be
% effective. It is even more difficult to develop automatic specification and
% verification tools due to limitations like state explosion, undecidability,
% etc.}
