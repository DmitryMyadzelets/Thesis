% Tesi di Laurea 
% Michele Furci
%\documentclass[final, 12pt, a4paper, twoside, italian, indent]{itthesis}
\documentclass[final, 12pt, a4paper, twoside, italian, indent]{itthesis}
\usepackage[latin1]{inputenc}
\usepackage[T1]{fontenc}
\usepackage{float}
\usepackage{babel}
\usepackage{psfrag}
\usepackage{babel}
\usepackage{graphicx}
\usepackage{makeidx}
\usepackage{varioref}
%\usepackage[dvips, light, first, bottomafter]{draftcopy}
\usepackage{pslatex}    % per avere i fonts PostScript
\usepackage{amsmath}    % per \genfrac in UserGuide.tex
\usepackage{amssymb}
\newcommand{\floorfrac}[2]{\genfrac{\lfloor}{\rfloor}{}{}{#1}{#2}}

%ambiente definizione
\newtheorem{definizione}{Definizione}

%ambiente teorema
\newtheorem{teorema}{Teorema}

\makeindex

\newcommand{\Index}[1]{#1\index{#1}}
\newcommand{\Indextt}[1]{\texttt{#1}\index{#1@\texttt{#1}}}
\newcommand{\Indexsc}[1]{\textsc{#1}\index{#1@\textsc{#1}}}

\newcommand{\Indexii}[2]{#1 #2\index{#1!#2}}
\newcommand{\Indexttii}[2]{#1 %
  \texttt{#2}\index{#1!\texttt{#2}@\texttt{#2}}}
\newcommand{\Indexscii}[2]{#1 %
  \textsc{#2}\index{#1!\textsc{#2}@\textsc{#2}}}

\makeatletter
\newlength{\@entrylabel@length}
\newcommand{\entrylabel}[2][\bfseries]{%
  \settowidth{\@entrylabel@length}{{#1 #2}}%
  \ifdim \@entrylabel@length > \labelwidth
     \parbox[b]{\labelwidth}{%     label troppo lunga: descrizione a capo.
       \makebox[0pt][l]{#1 #2}\\}%
  \else
     {#1 #2}%                      label corta: descrizione in linea.
  \fi
  \hfil\relax%
}
\newenvironment{entry}[1][\bfseries]%
   {\if@inlabel \leavevmode\global \@inlabelfalse \fi
    \begin{list}{}{%
       \renewcommand{\makelabel}{\entrylabel[#1]}%
       \setlength{\labelwidth}{1.5cm}%
       \setlength{\leftmargin}{\labelwidth}
       \addtolength{\leftmargin}{\labelsep}}
       \addtolength{\itemsep}{0.5\baselineskip}
   }%
   {\end{list}}
\makeatother


%%%%%%%%%%%%%%%%%%%[ Logos ]%%%%%%%%%%%%%%%%%%%%%%%%%%
\newcommand{\MathWorks}{\textsc{The~MathWorks,~Inc.\ }}
\newcommand{\Matlab}{\textsc{Matlab}}
\newcommand{\Simulink}{\textsc{Simulink}}
\newcommand{\PS}{\textsc{PostScript}}
\newcommand{\Il}{\textsc{Adobe Illustrator 7.0}}
\newcommand{\Ad}{\textsc{Adobe Systems Inc.}}

\begin{document}
%%%%%%%%%%%%%%%%%%%%%[ Front matter ]%%%%%%%%%%%%%%%%%%%%
\frontmatter
\University{Bologna}
\Faculty{Ingegneria}
\LCourse{Ingegneria dell'Automazione}
\Course{Controlli Automatici}

\Title{%
  Controllo di un quadrotor e manovre spericolateeee yeah!
}

\Author{Michele Furci}
\ChairPerson{\makebox[0pt][r]{Chiar.mo~Prof.~Ing.~Lorenzo~Marconi}}
\CoChairPerson{
  \makebox[0pt][r]{Dott.~Ing.~Roberto~Naldi}\\
  {Dott.~Ing.~Marco~Gilioli}
}

\Session{Invernale}
\Years{2011/2012}

\maketitle

\thispagestyle{empty}

%
% Parole chiave
%
\begin{entry}    %keywords
    \item[Parole chiave:] Quadrirotore\\ Controllo Automatico\\ Controllo non Lineare\\
    Controllo d'Assetto\\UAV\\
\end{entry}

\vfill

%
% Copyright e altre informazioni
%
\begin{flushleft}     %universit�, casy etc
\small C.A.S.Y ,Center for Research on Complex Automated Systems.\\
\small D.E.I.S., Dipartimento di Elettronica, Informatica e
Sistemistica.\\
\small Universit� di Bologna.

Le simulazioni sono effettuate in \Matlab{} e \Simulink{},
\Matlab{} e \Simulink{} sono marchi registrati di \MathWorks{}\\

La tesi � scritta in \LaTeXe, la stampa � in \PS.\\

\end{flushleft}

\Dedication{\normalsize Alla mia famiglia}

%\tableofcontents \listoffigures

\clearpage
%%%%%%%%%%%%%%%%%%%%%%%%%%%%%%%%%%%%%%%%%%%%%%%%%%%%%%%%%%

%
%..........
%
\chapter*{Ringraziamenti}
\vspace{0.5cm}\begin{flushright} {\em Bologna, 20 Dicembre 2012 }
\end{flushright}
{\em Desidero innanzitutto ringraziare la mia famiglia ...bla bla bla RINGRAZIAMENTI
} \vspace{0.5cm}
\begin{flushright}
{\em GRAZIE A TUTTI. In particolare a Gildo\\
Michele}
\end{flushright}
\clearpage
%%%%%%%%%%%[ Main matter: body of the document ]%%%%%%%%%%
\mainmatter
\chapter{Introduzione \label{ch:intro}}
    \input{Introduzione.tex}

\chapter{Quadrotor Overview \label{ch:quad}}
    \input{2_Overview.tex}

\chapter{Controllo \label{ch:control}}
    \input{5_Controllo.tex}

%\chapter{Pianificazione reattiva del moto}\label{ch:MotPlan}
 %   \input{MotionPlanning.tex}

%\chapter{Architettura del software di controllo}\label{ch:ArchSoft}
 %   \input{ArchitetturaSoftware.tex}

\appendix

%\chapter{La gestione dei vettori del piano in C++}\label{ch:App}
 %   \input{Appendice}

\addcontentsline{toc}{chapter}{Bibliografia}
\bibliographystyle{plain}
\bibliography{Bibliografia}
\nocite{*}

\backmatter \cleardoublepage \printindex

\end{document}
